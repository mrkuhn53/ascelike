%
%\documentclass[Journal]{ascelike}
 \documentclass[Proceedings]{ascelike}
%\documentclass[Preprint,10pt]{ascelike}
%
% Jan. 30, 2025
%
% include this line for producing ASCE-sytle citations and bibliography
\usepackage{biblatex}
% !!! Note that you should use "biber", not "bibtex", to process the citations.
%
% state your bibliographic ".bib" databases here.  Include the suffix ".bib".
\addbibresource{ascexmpl.bib}
%
% the following are some useful packages, but it's your choice
%
% handling graphics in figures
\usepackage{graphicx}
%
% AMS macros and fonts
\usepackage{amsmath}
\usepackage{amsfonts}
\usepackage{amsbsy}
%
% the booktabs package, for nicer tables
\usepackage{booktabs}
  \setlength\heavyrulewidth{1.5pt}
  \setlength\lightrulewidth{1.0pt}
  \setlength\abovetopsep{0pt}
  \setlength\belowrulesep{0pt}
%
% to use hyperlinks within pdf document (with pdflatex, not latex):
%\usepackage[colorlinks=true,citecolor=red,linkcolor=black]{hyperref}
%
% NOTE: Don't include the \NameTag{<your name>} if you have selected
%       the NoPageNumbers option: this leads to an inconsistency and
%       a warning, and the NameTag is ignored.
\NameTag{Kuhn, \today}
%
%
\begin{document}
%
% with Journal, manually make first letters uppercase
\title{Style Files for ASCE-like Documents~--- Version 3.0}
%
% enter authors and affiliations in the manner of the authblk.sty package.
% Authors' [*] are matched with affiliations' [*]
\author[1]{Matthew R. Kuhn, M.ASCE}
\author[2]{Second Author}
\author[2]{Third Author}
\affil[1]{Donald P. Shiley School of Engrg.,
          Univ.\ of Portland, 5000 N.\ Willamette Blvd.,
          Portland, OR 97203. E-mail: kuhn@up.edu.}
\affil[2]{Affiliation of the second and third authors.}
%
% you must process the title and authors
\maketitle
%
\begin{abstract}
Package \texttt{ascelike} produces manuscripts
that roughly comply with
guidelines of the American Society of Civil Engineers (ASCE).
The package has the options of producing journal submissions
for review (option \texttt{Journal}),
of producing ``camera ready'' submissions for ASCE's proceedings
publications (option \texttt{Proceedings}),
and of producing two-column preprints (option \texttt{Preprint}).
You are reading a document that was produced with the input
file \texttt{ascexmpl.tex}
and with the three components of
the \texttt{ascelike} package:
the document-class file \texttt{ascelike.cls}, 
the bibliographic style \texttt{ascelike.bbx}, and
the citation style \texttt{ascelike.cbx}.
This document serves as a brief guide to \texttt{ascelike},
as well as a test of its output and a template for your own papers.
The package is freely available from the CTAN archive
\verb+https://www.ctan.org+
and from \verb+https://github.com/mrkuhn53/ascelike+,
under the LaTeX 
Project Public License, version 1.3 or later.
%
% using the new command \KeyWords{}. For fidelity with ASCE journals, begin
% keywords with an uppercase letter, and separate keywords with semi-colons ;
\KeyWords{\LaTeX; ASCE; Document class; 
          ascelike.cls~(version 3.0);
          ascelike.bbx~(version 3.0);
          ascelike.cbx~(version 3.0).}
%
\end{abstract}
%
% for the Preprint option, the next line creates two-column documents.
% You can leave the line here, even if you don't use the Preprint option
\ifthenelse{\boolean{Preprint}}{\begin{multicols}{2}}{}
%
% this begins the document's text
\section{Introduction}\label{sec:Introduction}
%
Package \texttt{ascelike} produces
manuscripts that roughly comply with
guidelines of the American Society of Civil Engineers (ASCE).
The package consists of
the \texttt{ascelike.cls} document-class; the 
bibliographic and citation styles, \texttt{ascelike.bbx}
and \texttt{ascelike.cbx}; and the
example file \texttt{ascexmpl.tex}. 
All are available from the \texttt{CTAN} website
\verb+https://www.ctan.org+
and from \verb+https://github.com/mrkuhn53/ascelike+
\cite{Kuhn:2011a}.
These files are included in
distributions of \LaTeX, which are
readily available for most computer systems
(e.g. \textsf{TexLive}, \textsf{MikTex}, and \textsf{MacTex}).
ASCE has significantly changed its formats
since the earlier version of \texttt{ascelike} was created in 2010,
and \texttt{ascelike} has been updated accordingly.
The current version of \texttt{ascelike} is 3.0.
If you follow the instructions in this document,
but you are using an earlier version of \texttt{ascelike},
you will probably
encounter errors when trying to run \LaTeX\ with your document.
%
\par
The document that you are reading was created with the
\texttt{ascelike} package and with the input (template) file
\texttt{ascexmpl.tex}. 
This file also
serves as a test of
the \LaTeX\ \texttt{ascelike} package and as a template
for users' papers.
%
\par
Most persons access \LaTeX\ either through an online
cloud service (\texttt{CoCalc.com}, \texttt{Overleaf.com}, etc.)
or with a \LaTeX\ system installed on their computer.
If the former, the online service will likely already
have updated to version 3.0.
If \LaTeX\ is installed on your computer,
unless you have recently updated your distribution of 
\LaTeX\ (e.g. \textsf{TexLive}, \textsf{MikTex}, and \textsf{MacTex}),
your version of \texttt{ascelike} will likely be an earlier one.
The \texttt{ascelike} package should eventually be incorporated in
your computer's \LaTeX\ distribution,
and updating your distribution will activate the new
\texttt{ascelike}.
Until then, you can
download and install the most recent files from
\verb+http://www.ctan.org+.
For example, the three files,
\texttt{ascelike.cls}, \texttt{ascelike.bbx}, and \texttt{ascelike.cbx}
can be installed in the same folder as your document's
``\texttt{.tex}'' file before running \LaTeX.
%
\par
In addition to \texttt{ascelike.cls}, 
the files \texttt{ascelike.bbx} and \texttt{ascelike.cbx}
are used with
the bibliographic tool \textsc{biber} and the
package \textsc{Bib}\LaTeX\ to produce ASCE-like 
citations and bibliographic entries (with ASCE's use of
quotation marks around titles, etc.) \cite{Kuhn:2011a}.
Note that the package \texttt{biblatex} and the helper program 
\textsc{biber} should now be used, instead of the older
\textsc{bibtex}.
Likewise, the older \texttt{ascelike.bst} is deprecated by
the new files \texttt{ascelike.bbx} and \texttt{ascelike.cbx}.
An example bibliographic data base is given in the
supplementary file \texttt{ascexmpl.bib}.
\par
The document-class \texttt{ascelike.cls} requires several
supplementary packages:
\texttt{setspace}, 
\texttt{endfloat},
\texttt{lineno},
\texttt{authblk}, etc.
Without these packages, \texttt{ascelike} \emph{will not work},
and an error message will indicate the \texttt{*.sty} files that are missing.
These files are typically included with 
online cloud platforms (\verb+CoCalc.com+, \texttt{Overleaf.com}, etc.)
and in
\LaTeX\ computer distributions, such as
\textsf{TexLive}, \textsf{MikTex}, and \textsf{MacTex}.
All of the \texttt{ascelike} files are freely available
from the Comprehensive \TeX\ Archive Network (CTAN) archive,
\verb+http://www.ctan.org+.
If one of these files is not installed as part of your \TeX\ system,
you should download the file from the CTAN archive and place it in the same folder
as your manuscript files.
On GNU/Linux systems, these files are typically bundled
into software packages that can be downloaded and installed with
your system's package management utilities.
%
\par
In addition to these essential files,
we have found the following packages useful:
%
\begin{itemize}
\item
\texttt{graphicx} and its companion files for incorporating
EPS and PDF graphic files,
\item
\texttt{amsmath} and its companion files for AMS math formatting,
\item
\texttt{booktabs} to produce publication-grade tables, and
\item
\texttt{microtype} for tight kerning, as in ASCE's publications.
\end{itemize}
%
These packages 
are included in most \LaTeX\ distributions and
are freely 
available from the CTAN archive.
%
\par
The \texttt{ascelike} package is distributed under the
terms of the LaTeX Project Public License
(available from the CTAN archives),
either version 1.3 of the License, or any later version.
If you modify
\texttt{ascelike.cls}, you should rename it so that ``altered''
copies are not later proliferated.
Although \texttt{ascelike} is \emph{not} 
produced by ASCE, its agents,
or employees, \texttt{ascelike} is now referenced on
the ASCE web-site.
%
\section{Input Options}\label{sec:Input}
You should
prepare your \verb+*.tex+ input file as a regular
\LaTeX\ file,
but, of course, substituting ``\texttt{ascelike}'' for ``\texttt{article}''
in the opening \verb+\documentclass+ command.
The document class \texttt{ascelike.cls} provides the options given
below.
Choice of the main \verb+Proceedings|+\-\verb+Journal|+\-\verb+Preprint+ 
option is the most important; whereas,
the other options allow variations of the main option.
%
\begin{enumerate}
\item
The main option
\verb+Journal|+\verb+Proceedings|+\verb+Preprint+ specifies the overall 
format of the manuscript.
For most manuscripts, this is the only option needed;
the other options simply modify the main option.
%
\begin{itemize}
\item
\texttt{Journal} is the default format, 
producing manuscripts intended
for paper submissions to ASCE journals for review.
The default settings of \texttt{Journal}
are 12pt text, double-spaced text, numbered lines, and in-text figures.
These default settings can be altered with the options that are 
described below.
%
``\texttt{Journal}'' produces 
proper headings for
sections, subsections, subsubsections, appendices, and the abstract,
and it produces the proper page margins and numbers the pages.
Note that ASCE journals do not use numbered sections, subsections,
or subsubsections.
This behavior can be changed with the \texttt{SectionNumbers} option.
%
\item
``\texttt{Proceedings}'' produces
manuscripts for ASCE conference proceedings.  
As default settings, 
\verb+Proceedings+ uses single-spaced 12pt text,
places figures and tables within the text, and does not number lines.  
All of these default settings can be altered with the options that are
described below.
\texttt{Proceedings} produces proper headings for
sections, subsections, subsubsections, appendices, and the abstract.
%
\item
``\texttt{Preprint}'' produces
manuscripts that resemble the ASCE journal papers:
two-column output; 10~point fonts;
a one-column title, author block, and abstract;
and horizontal rules placed around the abstract and under figures.
Even with this option, however, some intervention is required 
by the user in their
\texttt{*.tex} document to achieve these effects.
These interventions are described below in the
section titled ``Preprints'' and are indicated within the
example file \texttt{ascexmpl.tex}.
\end{itemize}
%
\item
Option \verb+BackFigs+ can be used to override 
the default placing of tables within the manuscript,
so that figures and tables are placed at the end of document.
With \verb+BackFigs+, do not place any characters after
\verb+\end{figure}+ and \verb+\end{table}+ commands.
%
\item
Options \verb+SingleSpace|DoubleSpace+ can be used to override 
the default text spacing in the 
\texttt{Journal} and \texttt{Proceedings} formats.
%
\item
Options \verb+10pt|11pt|12pt+ can be used to override the 
default text size (12pt).
%
\item
The option \texttt{NoLists} suppresses inclusion of lists of tables
and figures, when \texttt{BackFigs} is also given as an option.
%
\item
The option \texttt{NoPageNumbers} suppresses the printing of page numbers.
%
\item
The options \verb+NoLineNumbers|LineNumbers+ can be used to override
the default use (or absence) of line numbers in the \texttt{Journal} and
\texttt{Proceedings} formats.
When \verb+LineNumbers+ is used with \texttt{Proceedings},
running \LaTeX\ throws a non-critical error message that can be
ignored to complete compiling the document.
%
\item
The option \texttt{SectionNumbers} produces an automatic numbering of sections.
Without the \texttt{SectionNumbers} option, sections will \emph{not} be
numbered, as this seems is the prefered formatting of ASCE publications 
(note that appendices will, however, be automatically
``numbered'' with Roman numerals).  
With the \texttt{SectionNumbers} option, sections and
subsections are numbered with Arabic numerals (e.g. 2, 2.1, etc.), but
subsubsection headings will not be numbered.  To change this default
depth of numbering when
the option \texttt{SectionNumbers} is invoked, 
insert the \texttt{$\backslash$setcounter$\{$secnumdepth$\}$} command
in the preamble of your document.
Even with the \texttt{SectionNumbers} option, you can use the ``starred''
form, \verb!\section*{ }!, to create a section heading without numbers.
This might be desirable for an Acknowledgements section at the end of
a paper.  Note, however, that the starred form will not suppress
the numbering of subsections or subsubsections.
\end{enumerate}
%
In addition, \texttt{ascelike} provides the new command \verb+KeyWords+,
described below.
%
\section{Preliminaries}\label{sec:prelim}
Version~3.0
diverges from previous versions:
the formatting of citations and bibliographic entries is now managed
by the \texttt{biblatex} package (not \textsc{bibtex}),
and this formatting is now processed with the \texttt{biber}
helper program, rather than with \texttt{bibtex}.
Unfortunately, \texttt{biblatex} cannot be loaded by the document-class
file \texttt{ascelike.cls}, and it must be 
manually loaded within your document.
In the documents preamble and immediately following the opening statement
\verb+\documentclass[...]{ascelike}+, you must include this one
line:
\begin{quote}
\verb+\usepackage{biblatex}+
\end{quote}
Again, you should use \textsc{biber} to process the document's citations,
not \textsc{bibtex}.
The \texttt{ascelike} document-class automatically passes the following
options to \texttt{biblatex}:
options \texttt{backend=biber, uniquename=init, style=ascelike}.
%
\section{Title, Authors, Abstract, and Keywords}\label{sec:title}
%
These aspects of version~3.0 are quite different from previous versions.
The title is entered, as before, with the standard \LaTeX\ command
\texttt{title$\{...\}$}.
\par
The package \texttt{authblk}
is now used for formatting the author block.
As described in the \texttt{authblk} documentation, the authors and
affiliations are input in the following manner:
\begin{quote}
  \verb+\author[1]{+\emph{author1}\verb+}+\\
  \verb+\author[1]{+\emph{author2}\verb+}+\\
  \verb+\author[2]{+\emph{author3}\verb+}+\\
  \verb+\author[2]{+\emph{author4}\verb+}+\\
  \verb+\affil[1]{+\emph{affil1}\verb+}+\\
  \verb+\affil[2]{+\emph{affil2}\verb+}+\\
  $\cdots$
\end{quote}
%
\par
The title and author block are followed by the abstract,
which should be
placed within the \verb+\begin{abstract}+~$...$ \verb+\end{abstract}+
environment.
%
\par
The command \verb+\KeyWords{<your key words>}+ can be used to produce
a labeled list of key words.  
Although it can be placed anywhere in the document,
if placed inside of the \texttt{abstract} environment,
it works nicely with the \texttt{preprint} option.
%
\par
In lieu of the \texttt{authblk} construct described above,
the standard \LaTeX\ commands of \verb+\author+
and \verb+\thanks+ can be used,
which will cause the affiliations to be appear as footnotes.
This alternative maintains backward compatability with \texttt{*.tex} files
created for older versions of \texttt{ascelike}.
%
\section{Sections, Subsections, Equations, etc.}\label{sec:Sections}
Sections and equations are entered in the usual manner,
and \texttt{ascelike} simply formats them in the ASCE styles.
For example,
section heading are automatically made uppercase with
\texttt{Proceedings} manuscripts, the first paragraph of sections
are not indented in \texttt{Journal} manuscripts, etc.
%
\subsection{An Example Subsection}\label{sec:Example}
%
No automatic capitalization occurs with subsection headings; 
you will need to capitalize the first letter of each word,
as in ``An Example Subsection.''
%
\subsubsection{An example subsubsection}%
\label{sec:Example2}
No automatic capitalization occurs with subsubsections; 
you will need to capitalize only the first letter of subsubsection 
headings.
\par
And now we include an example of a displayed equation (Eq.~\ref{eq:Einstein})
%
\begin{equation} \label{eq:Einstein}
E = m c^{2}
\end{equation}
%
a figure (Fig.~\ref{fig:box_fig}),%
%
\begin{figure}
%\begin{figure*}  % with Preprint option, use this one for two-column figures
  \centering
  \framebox[3.00in]{\rule[0in]{0in}{1.00in}}
  \caption{An example figure (just a box).}
  \label{fig:box_fig}
%
\end{figure}
%\end{figure*}  % with Preprint option, use this one for two-column figures
%
and a table (Table~\ref{table:assembly}).
%
\begin{table}
%\begin{table*}   % with Preprint option, use this one for 2-column tables
  \caption{An example table using the booktabs package}
  \label{table:assembly}
  \centering
  \small
  \begin{tabular}{ll}
    \toprule
    Assembly attribute & Values \\
    \midrule
    Number of particles & 4008 \\
    Particle sizes & Multiple  \\
    Particle size range & $0.45D_{50}^{\:\ast}$ to $1.40D_{50}$ \\
    Initial void ratio, $e_{\mathrm{init}}$ & $0.179$ \\
    Assembly size$^{\ast}$ & $54D_{50} \times 54D_{50} \times 54D_{50}$ \\
    \bottomrule
    \multicolumn{2}{l}{$^\ast$$D_{50}$ represents the median particle diameter}
  \end{tabular}
\end{table}
%\end{table*} % with Preprint option, use this one for two-column tables
%
\section{Citations and Bibliographic Entries}\label{sec:bibliogrphy}
When used together, the three components
of \texttt{ascelike}~--- \texttt{ascelike.cls}, \texttt{ascelike.bbx},
\texttt{ascelike.cbx}~---
use your \texttt{*.bib} bibliographic database file to
produce citations in name--year format.
In addition to including the statement
\verb+\texttt{authblk}+ in your document's preamble
(as instructed in the section titled Preliminaries),
\texttt{biblatex}
requires your document include two additional lines.
First, within your document's preamble
(i.e., prior to the \verb+\begin{document}+ statement),
you should provide the name of your input database, for example,
%
\begin{quote}
  \verb+\addbibresource{MyBibFile.bib}+
\end{quote}
%
Note that the ``\texttt{.bib}'' suffix must be included.
Second, at the location of your References section
(near the end of your document),
use the following statement to locate the section.
%
\begin{quote}
  \verb+\printbibliography+
\end{quote}
%
The above two statements are used \emph{in lieu of} the \textsc{bibtex}
statements \verb+\bibliography+,
\verb+\bibliographystyle+, and \verb+\thebibliography+.
%
\par
Unlike previous versions, by using the package \texttt{biblatex},
you should now process the bibliographic
entries with the helper program \textsc{biber}, \emph{not} \textsc{bibtex}.
The typical command sequence for processing your \texttt{*.tex} file 
to create a \text{PDF} file is as follows:
%
  \begin{quote}
    \verb+pdflatex <+\emph{your *.tex file prefix}\verb+>+\\
    \verb+biber <+\emph{your *.tex file prefix}\verb+>+\\
    \verb+pdflatex <+\emph{your *.tex file prefix}\verb+>+
  \end{quote}
%
although extra runs of \textsf{pdflatex} might be needed when figures
are placed at the end of the manuscript
(option \texttt{BackFigs}).
%
\par
As with previous versions,
the following citation options are available:
\begin{itemize}
\item
\verb+\cite{key}+ produces citations with full author 
list and year \cite{Ireland:1954a,Gaspar:2001a}.
\item
\verb+\citeA{key}+ produces citations with only the full 
author list: e.g. \citeA{Ireland:1954a}
\item
\verb+\citeN{key}+ produces citations with the full author list and year, but
which can be used as nouns in a sentence; no parentheses appear around
the author names, but only around the year: e.g. \citeN{Ireland:1954a}
states that \ldots
\item
\verb+\citeyear{key}+ produces the year information only, within parentheses,
as in \citeyear{Ireland:1954a}.
\end{itemize}
%
\par
Notwithstanding the above instructions,
you can also use the many citation options of
the \texttt{biblatex} package:
\verb+\parencite+, \verb+\textcite+, \verb+\smartcite+, etc.
%
Note that version~3.0 is backward-compatible 
with earlier versions:
you can still process the bibliography by \emph{not} using
\texttt{biblatex}, its commands \verb+\addbibresource{}+
and \verb+\printbibliography+, and \textsc{biber}.
This is done with the \LaTeX\ command \verb+\bibliography{}+
and using \textsc{bibtex}.
%
\par
The bibliographic data base \texttt{ascexmpl.bib}
gives examples of bibliographic entries for different document types.
The References section of this document is created with
the \texttt{ascelike.bbx} style for the following entries:
\begin{itemize}
\item journal articles \cite{Stahl:2004a,Pennoni:1992a},
\item a book \cite{Goossens:1994a,Evans:2003a}
\item an article in an edited book using 
      \texttt{@INCOLLECTION} \cite{Zadeh:1981a}
\item proceedings papers, using \texttt{@INPROCEEDINGS} 
      \cite{Eshenaur:1991a,Garrett:2003a}
\item a website using \texttt{@ONLINE},
      with the accessed date given in the \texttt{NOTE} field,
      and with the full ``Arizona Dept. of Commerce'' 
      given in the \texttt{NAME} field
      inside of braces $\{\:\}$
      \cite{Arizona:2005a,Foucher:2017a}
\item a masters thesis using \texttt{@MASTERSTHESIS} \cite{Sotiropulos:1991a}
\item a doctoral thesis using \texttt{@PHDTHESIS} \cite{Chang:1987a}
\item a forthcoming article (e.g., in press), with the field
     \texttt{pubstate = "forthcoming"}
     (note the lowercase "forthcoming", and quotation marks,
     not braces) \cite{Dasgupta:2008a,Han}
\item an anonymous book \cite{Moody:1988a},
\item an anonymous report using \texttt{@MANUAL} \cite{FHWA:1991a}
\item data sets using \texttt{@ARTICLE} and with the accessed date in the
      \texttt{NOTE} field \cite{Ansolabehere:2014a,Thernstrom:1986a}
\item a building code using \texttt{@MANUAL} or \texttt{@BOOK} \cite{ACI:1989a}
\item a discussion of an \texttt{@ARTICLE} \cite{Vesilind:1992a}
\item a paper in a foreign journal \cite{Ireland:1954a}
\item a standard using \texttt{@INCOLLECTION} \cite{ASTM:1991a}
\item a translated book \cite{Melan:1913a}
\item a two-part paper \cite{Frater:1992a,Frater:1992b}
\item a university report, using \texttt{@TECHREPORT} \cite{Duan:1990a}
\item an untitled item in the Federal Register using 
      \texttt{@MANUAL} \cite{FR:1968a}
\item a newspaper article using \texttt{@ARTICLE} and with the date
      given in the entry field \texttt{NOTE} \cite{Mossberg:1993a}
\item works in a foreign language \cite{Duvant:1972a,Reiffenstuhl:1982a}
\item electronic material as a CD, using \texttt{@ARTICLE} with
      the entry fields \texttt{JOURNAL} and \texttt{NOTE} \cite{Liggett:1998a}
\item software using \texttt{@MANUAL} \cite{Lotus:1985a}
\item two works by the same author in the same year
      \cite{Gaspar:2001b,Gaspar:2001a}
\item two works by three authors in the same year that only share
      the first two authors \cite{Huang2009a,Huang2009b}
\end{itemize}
%
\section{Creating Preprint Manuscripts}
A manuscript that approximates an ASCE journal paper can be created with
the following changes to your \texttt{*.tex} file:
%
\begin{enumerate}
\item
  Use the \verb+\documentclass[Preprint,10pt]{ascelike}+ at the
  start of your \texttt{*.tex} file.
\item
  Include the following command, which should follow the
  abstract and keywords, and should immediately precede
  the first section of text:
  \begin{quote}
    \verb+\begin{multicols}{2}+
  \end{quote}
\item
  Include the following command, immediately before the
  \verb+\end{document}+ at the end of the file:
  \begin{quote}
    \verb+\end{multicols}+
  \end{quote}
\item
  With each figure and table (i.e., float),
  you should decided whether it is a one-column or two-column float.
  For one-column floats, use the usual
  \verb+\begin{figure}+~$\ldots$ \verb+\end{figure}+
  and \verb+\begin{table}+~$\ldots$ \verb+\end{table}+ commands.
  For full-width floats that span two columns,
  replace these commands with their starred ``*'' versions:
  \verb+\begin{figure*}+~$\ldots$ \verb+\end{figure*}+
  and \verb+\begin{table*}+~$\ldots$ \verb+\end{table*}+.
\end{enumerate}
%
% uncomment the following line to begin the references on a new page
%\pagebreak
%
% if you have no appendices, comment or delete this line
\appendix
%
% an example of notation section, using the tabular environment. Note that 
% this and all appendices (except References) start with the \section command
%
\section{Notation}
\emph{The following symbols are used in this paper:}%\par\vspace{0.10in}
\nopagebreak
\par
\begin{tabular}{r  @{\hspace{1em}=\hspace{1em}}  l}
$D$                    & pile diameter (m); \\
$R$                    & distance (m);      and\\
$C_{\mathrm{Oh\;no!}}$ & fudge factor.
\end{tabular}
%
% Here, we use "\printbibliography" to create an "*.aux" file that can be
% processed with biber, using the stylefiles, ascelike.bbx and ascelike.cbx
\printbibliography
%
%
% the next line is for the Preprint option, for a two-column preprint. You
% can leave the line here, even if you don't use the Preprint option
\ifthenelse{\boolean{Preprint}}{\end{multicols}}{}
%
\end{document}
